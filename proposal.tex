\documentclass[12pt,a4paper]{article}
\usepackage[utf8]{inputenc}
\usepackage{hyperref}

\usepackage{harvard}
\citationmode{abbr}

\author{Matti Nelimarkka\footnote{Master of Social Sciences, University of Helsinki and Student of Sciences, University of Helsinki, \texttt{<\email{matti.nelimarkka@hiit.fi}>}}}
\title{PhD Thesis Proposal}
\begin{document}

\maketitle

\tableofcontents

\newpage

\section{Introduction}

\section{Theoretical background}

\subsection{Technology and power -- the existing literature}

\citeasnoun[•••]{winner85} has made an observation regarding city design, that highlights power from technical design aspect. He observes, that

\begin{quote}
anyone who has travelled the highways of America and has gotten used to the normal height of overpasses may well find something a little odd about some of the bridges over the park ways on Long Island, New York. Many of the overpasses are extraordinarily low, having as little as nine feet of clearance at the curb. Even those who happened to notice this structural peculiarity would not be inclined to attach any special meaning to it. In our accustomed way of looking at things such as roads and bridges, we see the details of form as innocuous and seldom give them a second thought.
\end{quote}

His explanation for this observation indicates that it is not a random effect but decision made by a person:

\begin{quote}
It turns out, however, that some two hundred or so low-hanging overpasses on Long Island are there for a reason. \texttt{- -} Robert Moses, the master builder of roads, parks, bridges, and other public works of the 1920s to the 1970s in New York, built his overpasses according to specifications that would discourage the presence of buses on his parkways.
\end{quote}

Based on this, Winner concludes that

\begin{quote}
[t]hey were deliberately designed and built that way by someone who wanted to achieve a particular social effect.
\end{quote}

In online participation environments some research has been done to examine the decision makers and their role. In institutional environments, role of administration has been studied \citeaffixed{nelimarkka11}{e.g.}

\subsection{Lukes's view on power}

\possessivecite{lukes05} classical view differentiates three distinct types of power,

\begin{description}
\item[direct power]
\item[indirect power]
\item[normative power]
\end{description}

\subsection{The three approaches to design research}

\subsection{e-participation}

\section{Research questions}

\subsection{How is power used in enabling and disabling e-participation?}

\subsection{How is the use of power explained and rationalized?}

\subsection{Exploration of the effects of the decisions}

In political science, experimental methods

\section{Building a research case}

\subsection{Methodologies}

\subsection{Data}

\bibliographystyle{apsr}
\bibliography{proposal}

\end{document}