% \documentclass[journal,a4paper]{IEEEtran}
\documentclass{article}
\usepackage[utf8]{inputenc}
\usepackage{hyperref}

\usepackage{harvard}
\citationmode{abbr}

\author{Matti Nelimarkka}
\title{Draft: Supporting Deliberative Discussion with Enhanced Design}

\begin{document}

\maketitle

%% nice output
\setlength{\parindent}{0pt}
\setlength{\parskip}{1ex}

\begin{abstract}
Online deliberation systems are well studied also from the domain of political science. The results however are less encouraging: both in terms of quality of the participation and reach and wide online deliberation has not achieved the ideals set up by political scientists and practitioners. Even while these systems are well studied, the focus has been in the use of these systems and in lesser degree in development-phase. However, the design of these systems impacts the user behavior and therefore needs more rigorous analysis.

This analysis requires skills from both human-computer interaction (HCI) and political science: the questions related to system design and impacts of them are well known in HCI, where as in the concepts used to frame and analyze this question must be taken from political science. A classical HCI approach applies experimental methods to compare different interfaces. I argue that this method can be applied in online deliberation domain, and furthermore, by using this approach set of design principles can presented for practitioners and for scholars a framework for analysis and examination on the impacts -- and interplay -- of social structures and technical systems on deliberation.

The first analysis focus is the \textit{the quantity of participation}. The focus should be both in the uptake and long-term use of the system. The research tradition in this area has focused in examining impact of background variables into participation, such as socio-economical factors. This research proposal aims to study not only those, but variations of the software tool impacting participation, such as anonymity.

The second and third analysis foci are related \textit{the quality of participation}. This includes both the more objective quality -- which can be estimated with e.g. \possessivecite{steenbergen03} discourse quality index (DQI) -- and the subjective experience of participation. Here the traditions are mixed, the first research aims to quantify qualitative material and use statistical methods to understand it, where as the second method is more classical qualitative study, aiming to understand how participants feel about the participation, e.g. via interviews. The former is classically used in online participation research where as the latter, more user experience-focus question is not commonly used, but is a vital part of the success of participation.
\end{abstract}

\newpage

\section{Introduction}

Scholars of political thought, such as \citeasnoun{habermas87}, \citeasnoun{fiskin91} and \citeasnoun{cohen97}, have suggested that the decisions made in our society could be improved by arguing and discussing them in an open and equal environment, which could be called \textit{the public sphere}. This normative standing is called deliberative democracy, and the concept has been studied in detail both by the normative and empirical political scientists \citeaffixed{xx,yy,xx,yy}{e.g.}. Compared to classical representational democracy, deliberative democracy is direct. In representative democracy, the discussion before decision-making is not mandatory and decision are dominated by the majority group. The deliberative approach, as stated above, highlights discussion before decisions are made and the exact decision-making mechanics may vary, ranging from consensus to deliberative poll. A fundamental requirement for deliberative democracy is a forum for discussion, which may be a town hall meeting, a mini-public or, thanks to the rise of the Internet, an online discussion system \citeaffixed{smith09}{e.g.}.

Online participation is a widely studied topic, and facilitating discussions -- the deliberative standing point -- is just one of the fields in focus \citeaffixed{dahlberg11}{c.f.}. The research has focused on studying how political discussion takes place in venues of different kind \citeaffixed{macintosh02,jen03}{e.g.} and evaluated these discussions \citeaffixed{strandberg08,dahlberg01,albrecht06,chadwick03}{e.g.}. Online deliberation research has extended also in non-political forums as a place for discussion \cite{graham12}. This extension indicates the social importance to study the online discussion forms in detail, as they are not only in focus of politics, but in general communication. However, less scholarly focus has been on the design of the discussion forums and the implications of design on the participation \citeaffixed{wright07,sukumaran11}{however, e.g.}.

This research proposal suggests that studying this design in more detailed manner will support development of online forums. I aim to study, how the design, both technical capabilities of the system combined with the social environment, influence the deliberation and discussion on the system. This research combines skills and methods from human-computer interaction (HCI) with the concepts of social science, and especially in deliberative democracy discussed above. I argue that use of experimental method and field trials allows empirically validating the impacts of the design to the deliberation. This empirical and evidence based validation will provide stronger point of argumentation than the current research in this domain. Use of experimentation and field trials is a standard procedure in HCI \citeaffixed{brown11}{e.g.}, but less used in political science currently \citeaffixed{green03,stoker10}{e.g.}. Oxford-research?

Metatekstiä?

\section{Research questions}

As noted above, the design of an online forum is not extensively studied. However, scholars such as \citeasnoun{wright12}, argue that the process side should be further studied, however the design studies and empirical studies dominating the study of online forums do not have strong experimental background, which leads to less rigorous analysis of this problem area. This indicates a research gab both as a research area and as a methodological approach.

My core argumentation is that the participation system impacts the resulting participation. \citeasnoun{stromer-galley04} suggests terms '\textit{interactivity-as-product}' and '\textit{interactivity-as-process}' to separate these distinct problems. The participation tool as a product is designed and build to certain uses. The system design should take in account both the social context of use and the technical affordances, limitations and feedback mechanics the system provides, as they all play an significant role in the system adaptation and use \citeaffixed{sukumaran11,herring07,underhill03}{e.g.}. Examples of design choices impacting the social context could be a choice of moderation strategy, where as more technical choices could be a choice if the system uses pseudonyms or real identities. As we can from these examples see, these factors can be strongly linked together: allowing the system to be anonymous naturally impacts the social context of use, even while it is more a technical factor.

The exact research questions suggested in this proposal combine the two aspects of participation: quantity and quality \citeaffixed{smith09}{e.g.} together with the system design exploration. Research questions investigate how the system design -- considered in broad sense, as explained above -- impacts the participation. In detail, the aim is to study

\begin{enumerate}
\item how does the design affect uptake and use of the participation system?
\item how does the design impact the quality and rationality of the deliberation?
\item how does the design impact how participant experiences deliberativeness?
\end{enumerate}

\section{Approach}

As explained above, this research proposal aims to study how design choices impact the deliberativeness of online forum. As suggested by \citeasnoun{smith09}, the quality of deliberation can be determined by examining who participate, how they participate and what is the impact of participation. Measurements of these will be the dependent variables. There are a variety of potential interventions that can be made, and they will be a major part of the experimental design. Examples of potential interventions are listed in Table \ref{tab:modalities}.

\begin{table}
\begin{tabular}{lc|l}
Social structure & & \cite{sukumaran11,underhill03} \\
Affordances \& limitations & & \cite{eklundh94,danet98,donath99} \\ 
Feedback mechanics & &
\end{tabular} 
\caption{Possible modalities and related research}
\label{tab:modalities}
\end{table}

Experimenting these conditions in an online forum will enable me to study deliberation and impact of the design choices in detail. Even while experimental methods in political science are not mainstream at this time, the need for comparing different situations and providing support for applied policy has been identified \cite{green03}. \citeasnoun{stoker10} discusses in great length about using the skills and the knowledge of political scientists to support the policy design ongoing by conducting field trials. Also, in the area of online participation and discussion forums research, this approach has gained interests. \citeasnoun[256]{wright12} identifies this research opportunity, and one of the emerging research areas he highlights aims to

\begin{quote}
\texttt{[- -]} comparatively testing different forum interfaces to see how they impact deliberation (and other values) would enhance Saward’s democratic toolkit.
\end{quote}

Evaluating the users of online forums is not a novel research question as such. This research is a good frame, which can be used to analyze the content. One theme in previous research is the factors affecting participants' behavior and contribution. This research uses statistic methods, such as logistic regression, to examine the impact of these factors. Previous research into participation suggest that participation is unequal -- that there are differences e.g  based on socio-economic factor \citeaffixed{hindman09,albrecht06,strandberg08}{regarding online participation, e.g.}.

Also, different kind of content analysis is often conducted in online domain, using different kind of criteria to evaluate the content -- also in the area of deliberation \cite{x,y,z}. However, one of the problems in this research is the differences in operationalizations, which means that the research results can not easily be applied to other research subjects. This is why I suggest using  \possessivecite{steenbergen03} Discourse Quality Index (DQI) to measure the quality of the content. DQI measures how well discourses approach to the deliberative ideal. By using well defined and strict approach, I expect to increase validity of this research. However \possessivecite{steenbergen03} approach is locked to a certain view of deliberative democracy, which is criticized by \citeasnoun{freelon10} and \citeasnoun{coleman12}. They argue -- and I agree -- that these kind of measurement focus certain kind of constructed understanding of good discussion. However, they key reason to use also these kind of measures is to be able to compare results between services, and in those cases empirically grounded and documented tools, such as DQI, are good ones.

Lastly, an interesting view in \citeasnoun{baek11} is their approach also to consider the experience of the participants as part of deliberation \citeaffixed{coleman12}{also highlighted in}. They argue that the deliberation depends on the perceived heterogenity, not only about the true heterogenity of the population. Even while they do not make the clear link, I see the percived experience related to the social context discussed above. Also, in general focus on experience is seen important in modern HCI research, where the user experience is seen to impact product's success. Therefore, studies should also take account this kind of subjective feeling.

\newpage

\bibliographystyle{apsr}
\bibliography{proposal}

\end{document}
