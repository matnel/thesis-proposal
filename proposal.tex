% \documentclass[journal,a4paper]{IEEEtran}
\documentclass{article}
\usepackage[utf8]{inputenc}
\usepackage{hyperref}

\usepackage{harvard}
\citationmode{abbr}

\author{Matti Nelimarkka}
\title{Draft: Supporting Deliberative Discussion with Enhanced Design}

\begin{document}

\maketitle

%% nice output
\setlength{\parindent}{0pt}
\setlength{\parskip}{1ex}

\begin{abstract}
Online deliberation systems are well studied also from the domain of political science. The results however are less encouraging: both in terms of quality of the participation and reach and wide online deliberation has not achieved the ideals set up by political scientists and practitioners. Even while these systems are well studied, the focus has been in the use of these systems and in lesser degree in development-phase. However, the design of these systems impacts the user behavior and therefore needs more rigorous analysis.

This analysis requires skills from both human-computer interaction (HCI) and political science: the questions related to system design and impacts of them are well known in HCI, where as in the concepts used to frame and analyze this question must be taken from political science. A classical HCI approach applies experimental methods to compare different interfaces. I argue that this method can be applied in online deliberation domain, and furthermore, by using this approach set of design principles can presented for practitioners and for scholars a framework for analysis and examination on the impacts -- and interplay -- of social structures and technical systems on deliberation.

The first analysis focus is the \textit{the quantity of participation}. The focus should be both in the uptake and long-term use of the system. The research tradition in this area has focused in examining impact of background variables into participation, such as socio-economical factors. This research proposal aims to study not only those, but variations of the software tool impacting participation, such as anonymity.

The second and third analysis foci are related \textit{the quality of participation}. This includes both the more objective quality -- which can be estimated with e.g. \possessivecite{steenbergen03} discourse quality index (DQI) -- and the subjective experience of participation. Here the traditions are mixed, the first research aims to quantify qualitative material and use statistical methods to understand it, where as the second method is more classical qualitative study, aiming to understand how participants feel about the participation, e.g. via interviews. The former is classically used in online participation research where as the latter, more user experience-focus question is not commonly used, but is a vital part of the success of participation.
\end{abstract}

\newpage

\section{Introduction}

Scholars of political thought, such as \citeasnoun{habermas87}, \citeasnoun{fiskin91} and \citeasnoun{cohen97}, have suggested that the decisions made in our society could be improved by arguing and discussing them in an open and equal environment, which could be called \textit{the public sphere}. This normative standing is called deliberative democracy, and the concept has been studied in detail both by the normative and empirical political scientists \citeaffixed{xx,yy,xx,yy}{e.g.}. Compared to classical representational democracy, deliberative democracy is direct. In representative democracy, the discussion before decision-making is not mandatory and decision are dominated by the majority group. The deliberative approach, as stated above, highlights discussion before decisions are made and the exact decision-making mechanics may vary, ranging from consensus to deliberative poll. A fundamental requirement for deliberative democracy is a forum for discussion, which may be a town hall meeting, a mini-public or, thanks to the rise of the Internet, an online discussion system \citeaffixed{smith09}{e.g.}.

Online participation is a widely studied topic, and facilitating discussions -- the deliberative standing point -- is just one of the fields in focus \citeaffixed{dahlberg11}{c.f.}. The research has focused on studying how political discussion takes place in venues of different kind \citeaffixed{macintosh02,jen03}{e.g.} and evaluated these discussions \citeaffixed{strandberg08,dahlberg01,albrecht06,chadwick03}{e.g.}. Online deliberation research has extended also in non-political forums as a place for discussion \cite{graham12}. This extension indicates the social importance to study the online discussion forms in detail, as they are not only in focus of politics, but in general communication. However, less scholarly focus has been on the design of the discussion forums and the implications of design on the participation \citeaffixed{wright07,sukumaran11}{however, e.g.}. This is surprising, as the most common observation in online deliberation research suggests that researchers' ideals of good discussion and deliberation have not been achieved: therefore, in the online domain, there is room for proposals that aim to improve these systems.

This research proposal suggests that studying this design in more detailed manner will support development of online forums and improve them. I aim to study, how the design, both technical capabilities of the system combined with the social environment, influence the deliberation and discussion on the system. This research combines skills and methods from human-computer interaction (HCI) with the concepts of social science, and especially in deliberative democracy discussed above. I argue that use of experimental method and field trials allows empirically validating the impacts of the design to the deliberation. This empirical and evidence based validation will provide stronger point of argumentation than the current research in this domain. Use of experimentation and field trials is a standard procedure in HCI \citeaffixed{brown11}{e.g.}, but less used in political science currently \citeaffixed{green03,stoker10}{e.g.}.

\section{Research statement}

Above a gab in the research was indicated. The empirical research in online deliberation has focused on behaviors of use, but it has not examined the systems as technical artifacts. However, scholars such as \citeasnoun{wright12}, argue that this process side should be further studied. The core argument in this work is that this design of the system and choices made in the design impacts the resulting participation. Said differently, I assume that in certain environments, certain design choices are preferable as they lead to improved participation, either in terms of quality or quantity.

My thinking evolves from \possessivecite{stromer-galley04} concepts '\textit{interactivity-as-product}' and '\textit{interactivity-as-process}'. Se makes a distinction between the technical apparatus and the participation as doing. The participation tool as a product is designed and build to certain uses. The system design should take in account both the social context of use and the technical affordances, limitations and feedback mechanics the system provides, as they all play an significant role in the system adaptation and use \citeaffixed{sukumaran11,herring07,underhill03}{e.g.}. Teknologiadeterminismi?

Some examples of design choices impacting the social context could be a choice of moderation strategy, where as more technical choices could be a choice if the system uses pseudonyms or real identities. As we can from these examples see, these factors can be strongly linked together: allowing the system to be anonymous naturally impacts the social context of use, even while it is more a technical factor.

As stated above, this research proposal aims to study the role of design in the participation. Anonymity and moderation strategy are two possible concepts related to those. In detail this research focuses on three different indicators to measure the impact, and therefore three research questions can be stated out:

\begin{enumerate}
\item How does the design affect uptake and use of the participation system?
\item How does the design impact the quality and rationality of the deliberation?
\item How the participants' experiences are affected by the system design?
\end{enumerate}

The first question evaluates the users of online forums. This is not a novel research question as such, but could be considered similar to classical election studies and has indeed been conducted in online forums too \citeaffixed{hindman09,albrecht06,strandberg08}{regarding online participation, e.g.}. Therefore, this question has a research tradition, and one theme in previous research is the factors, such as age, socio-economic status or political knowledge, affecting participants' behavior and contribution. This theme is studied by using statistical methods, such as logistic regression. My main contribution here is to attach control variables related to the participation systems' characteristics into these models and instead of examining the impact of these personal attributes to examine the impact of the system characteristics.

The second question studies the content produced by the users. Again, this is not a novel research theme for online deliberation study, different kind of content analysis are often conducted iusing different kind of criteria to evaluate the content \cite{pietila02,y,z}. However, one of the problems in this research is the differences in criteria and operationalizations, which means that the research results can not easily be applied to other research subjects. This is why I suggest using  \possessivecite{steenbergen03} Discourse Quality Index (DQI) to measure the quality of the content. DQI measures how well discourses approach to the deliberative ideal. By using well defined and strict approach, I expect to increase validity of this research. At the same time \possessivecite{steenbergen03} approach is locked to a certain view of deliberative democracy, which is criticized by \citeasnoun{freelon10} and \citeasnoun{coleman12}. They argue -- and I agree -- that these kind of measurement focus certain kind of constructed understanding of good discussion. However, they key reason to use also these kind of measures is to be able to compare results between services, and in those cases empirically grounded and documented tools, such as DQI, are good ones. The idea is therefore examine, if the system design choices, such as use of anonymity, lead to higher quality deliberation.

The last question focuses again in the users and in their experience of the system. As pointed out by \citeasnoun{baek11}, the experience of the participants should be used to evaluate the success of deliberation and discussion \citeaffixed{coleman12}{also highlighted in}. They argue that the deliberation depends on the perceived heterogeneity, not only about the true heterogeneity of the population. Even while they do not make the clear link, I see the perceived experience related to the social context discussed above. Also, in general focus on experience is seen important in modern HCI research, where the user experience is seen to impact product's success. Therefore, studies should also take account this kind of subjective feeling, and in the deliberative system design the aim should be development of good participation experience too, not only good quality of participation using researchers' objective measurements.

\section{Approach}

As explained above, this research proposal aims to study how design choices impact the deliberativeness of online forum. As discussed above, the concept of deliberativeness is divided to three domains: quantity, quality and experience. These will act as dependent variables, i.e. the impact of the design choices made will be examined using these three aspects.

The approach, as hinted above uses experimental approach and field trials to study the impact of single design choices. This approach is often used in HCI, and use of experimental methods is also known in political science \cite{brown11,green03,stoker10}. Also online forum research, the potential for this kind of approach has been detected: \citeasnoun{wright12} has suggested empirical testing as one of the future research needs in online domain

\begin{quote}
Put simply, e-democracy tools can themselves be viewed as discretely designed interfaces that can be embedded with specific institutional norms, values and procedures \cite{wright07}. Experimental research designs would allow these claims to be tested in depth.
\end{quote}

\begin{table}
\begin{tabular}{lc|l}
Social structure & & \cite{sukumaran11,underhill03} \\
Affordances \& limitations & & \cite{eklundh94,danet98,donath99} \\ 
Feedback mechanics & &
\end{tabular} 
\caption{Possible modalities and related research}
\label{tab:modalities}
\end{table}

Even while there is existing experimental research (see Table \ref{tab:modalities}), this far the results have not taken in account the different aspects nor discussed the topic especially using the deliberative approach. This has also been pointed in \possessivecite{wright12} work. However, as demonstrated in \citeasnoun{margetts09a} and \citeasnoun{margetts09b}, the experimental method can be used to examine interaction caused by different conditions. In this research proposal the conditions are related to the discussion system, e.g.:

\begin{itemize}
\item anonymity
\item possibility to vote messages up or down
\item limitations on message length
\end{itemize}

The data will be collected using three approaches: in certain cases a laboratory study is possible and can be used to examine the participation \citeaffixed{sukumaran11}{e.g.}, however more viable is  to set up a field trial environment for discussion and examine that discussion. For example, research could be conducted using the university's classes and the online support discussions in those as the main sample, this way the participants will have a motivation using the system, but the study is conducted in the wild. This requires strong collaboration with the organizers, but will ensure certain level of participation. Naturally, other organizations can also be used to conduct this study, however the organization should provide the motivation to use and participate the platform, so that the field research is not taking place in void. Lastly, after a change to discussion system the data before and after the change can be used to examine impact caused by the change \citeaffixed{kilner05}{e.g.}.

The former two approaches allow stronger control of the phenomena in question, such as anonymity. To keep the context similar, participants must be separated in condition groups, and these groups are displayed different content. For example, if participant is chosen in the anonymous condition, the participant will be able to discuss only with other people with the same condition: this way, there will not be contamination caused by the different conditions -- which would impact e.g. to the social context and as such could impact the results \cite{sukumaran11}. The argument in these cases is that other factors impacting participation can be stabilized, and therefore differences -- if any -- are caused by the phenomena studied.

The latter approach, using data of change in an existing system, is not as controlled: the difference may not be caused only by the change in the forum, but instead could be related to the time and seasonal differences. However, the benefit of this approach is the possibility to achieve high $n$, which in laboratory and field trials is harder to achieve, especially when the object of testing is an interactive system. However, when the sample is large enough, the impact of the phenomena studied should be visible in the data.

\section{Conclusions}

This research proposal has examined a gab in online deliberation research: even while several studies have examined the discussion and participation in these, the design choices and their impact on the discussion and participation have been studied less. However, if by adapting the system we can achieve the deliberative ideal, these changes can impact the utilization of online deliberation in the future. Especially, as in the modern era participation is vital part of the Web, not limited to political participation, the societal impact of online deliberation research is high.

I suggest that deliberative participation consists of quantity, quality and experience. These dimensions must be taken in account when evaluating the platform and changes made into it. For example, a research question for this study could be: "How did anonymity impact the perceived quality of the discussion?" -- highlighting the design choice (anonymity) and the dimension studied (experience).

Lastly, this research proposal and the aims suggest use of experimental approach: this way rigorous analysis of the impact of single design choices can be made. Experiments can be organized in a lab study form or in field trials, where I have strong preference to conduct field trials to gain better validity of the results. Also, potential data source are changes made in commenting systems e.g. in news papers web sites, where even while the environment can not be controlled by the researcher, the sample size are large. All of these data can be analyzed using the quantity, quality and experience -approach discussed above.

\newpage

\bibliographystyle{apsr}
\begin{small}
\bibliography{proposal}
\end{small}

\end{document}
