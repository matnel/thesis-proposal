\documentclass[journal,a4paper]{IEEEtran}
\usepackage[utf8]{inputenc}
\usepackage{hyperref}

\usepackage{harvard}
\citationmode{abbr}

\author{Matti Nelimarkka}
\title{Designed for Democracy}
\begin{document}

\maketitle

%% nice output
\setlength{\parindent}{0pt}
\setlength{\parskip}{1ex}


\section{Introduction}

The impact of the emerging technologies to the democracy has been widely studied topic. The rise of the Internet is again one media that provides potential for support of democracy and political participation. The use of the Internet has been studied to support the institutional participation, to support novel forms of social movements and to facilitate discussions, among others. Looking the discussion facilitation in detail and focusing deliberation in democratic process, different platforms have been experimented \citeaffixed{macintosh02,jen03}{e.g.} and evaluated \citeaffixed{strandberg08,albrecht06}{e.g.}.

Considering the recent developments in studying online discussion, \citeasnoun{graham12} suggest that the methodology can be extended beyond classical political discussion forums to other kind of discussions that facilitate social interaction. This observation, that I agree on, indicates also the societal impact of detail examination of online discussion forums: they cover more than the classical institutional politics: in \possessivecite{graham12} case, the focus of study was discussion related to a television program.

Based on the extensive research it is not clear weather the Internet supports deliberative democracy or not.

\section{Research questions}

This research proposal aims to study the impact design decisions have on participation. The suggestion is that certain decisions, such as anonymity and choices on moderation affect quality of deliberation. Previous research in this domain is narrow and does not have a strong empirical focus. \citeasnoun{fatland07} has designed a new interface for political participation from deliberative point of view, but it has not been evaluated in practice. \citeasnoun{wright07} \citeaffixed{wright06}{also discussed in} studied the impact of the moderation for the discussion. They suggest that moderation helps to keep the discussion on topic. This study however does not aim to explain the impact of other aspects, such as anonymity and profiles, to deliberation quality nor the quality of the deliberation been examined in detail.

\citeasnoun{wright07} however emphasize the need to study the process and use of online forums, especially as there may be power relations in the design process.  Some research has focused on understanding the design process and the stakeholders in the process. My previous work suggested that the existing organization culture impacts the implementation of online democracy \cite{nelimarkka11}, but the result is ot novel. 

\possessivecite{smith09} work on assessment of deliberative practices suggests, among others\footnote{Other aspects \possessivecite{smith09} suggest are transferability of the mechanics and impact of the citizens' participation. Naturally, they are important aspects and valuable further research questions.} that the both the quality of the discussion and diversity of participants should be evaluated. The quality of the discussion points to the area

Therefore the research questions must also cover both of these aspects, and 

\begin{enumerate}
\item How the design of the online participation affects the users?
\item How the design of the online participation impacts the quality of the discussion?
\end{enumerate}

\section{Approach}

Use of field experimentation is required ...

Evaluating the users of online forums is not a novel research question. Classical methods apply statistical methods, such as logistic regression is beneficial

For measuring the deliberative qualities of the discussion, one potential approach is to apply \possessivecite{steenbergen03} Discourse Quality Index (DQI). This measurement tool has been developed to ... Kritiikkiä??

\section{Random}

The work in this domain however should continue, and \citeasnoun[256]{wright12} suggest that

\begin{quote}
\texttt{[- -]} comparatively testing different forum interfaces to see how they impact deliberation (and other values) would 
enhance Saward’s democratic toolkit.
\end{quote}

\bibliographystyle{apsr}
\bibliography{proposal}

\end{document}