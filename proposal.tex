\documentclass[journal,a4paper]{IEEEtran}
%\documentclass{article}
\usepackage[utf8]{inputenc}
\usepackage{hyperref}

\usepackage{harvard}
\citationmode{abbr}

\author{Matti Nelimarkka}
\title{Draft: Designed for Deliberative Discussion}
\begin{document}

\maketitle

%% nice output
\setlength{\parindent}{0pt}
\setlength{\parskip}{1ex}

\section{Introduction}

Political science scholars have suggested that the decisions made in our society could be improved by arguing and discussing them before hand \cite{xxx}. This approach can be called as deliberative democracy, and the concept has been studied in detail both by the normative political scientists and the empirical political scientists\footnote{Summary of research}. Deliberative democracy requires a venue for discussion, which may be a town hall meeting, mini-publics or, thanks to the rise of the Internet, an online forum.

In political science e-participation is a widely studied topic and discussion facilitation is just one of the fields in focus\footnote{This choice reflects a deliberative view about democracy, however even in the field of e-participation there are different positions on the approaches to democracy. \citeasnoun{dahlberg11} distincts liberal-individualist, counter-public and autonomist marxist approaches to democracy as optional to the deliberative chosen in this work.} The research has focused on studying how political discussion takes place in different kind of venues \citeaffixed{macintosh02,jen03}{e.g.} and evaluated these discussion \citeaffixed{strandberg08,dahlberg01,albrecht06,chadwick03}{e.g.}. Recently, the discussion has also focus on non-political forums as places for deliberation, for example in \possessivecite{graham12} study, the discussion related to a television program was examined from deliberative point of view. This indicates the social importance to study the online discussion forms in detail.

However, less scholarly focus has been on the design of the discussion forums and the implications of design \citeaffixed{wright07}{however, e.g.}. Indeed, \citeasnoun{sukumaran11} suggest that the perceived social context impacts the thoughtfulness of discussion. This proposal suggest that these design choices and their impact on the user behaviour should be further studied in an experimental setting to understand how to design these forums in an improved way.

\textbf{Kappale tähän väliin design-tutkimuksesta?}

On the other hand the extensive research on online participation has not been as successful as one could hope. For example, \citeasnoun[142]{smith09} concludes that "whilst there have been staggering developments in the commercial work, the potential for using ICT to increase and deepen citizen participation in political decision-making has legged somewhat behind." In detail xx, yy and zz focus on these problems. \textbf{How much previous research should be examined?}

\section{Research questions}

This research proposal aims to study how design choices impact online forum's deliberativeness. As suggested by \citeasnoun{smith09}, quality of deliberation can be determined by examining who participate, how they participate and what is the impact of participation. These will be among the dependent variables. There are variety of potential variations (independent variables), that can be made. Some of these are listen in the table\ref{tab:modalities}.

However, \citeasnoun{wright07} arguest that the process side of online forums should be studied further. Certain steps have been made to progress into this direction. For example, \citeasnoun{jen03} and nn discuss specially the role of politicians as part of participation, and xx and yy discuss about administration's role.

Research on forums design includes \possessivecite{fatland07} novel interface for deliberative political participation -- but it has not been evaluated. Empirical evaluations are also lacking, for example \citeasnoun{wright07} studied the impact of moderation. They suggest that moderation helps to keep the discussion on topic, but does not discuss other factors and aspect affecting the participation, such as anonymity and profiles. And the quality of the deliberation has not been examined in detail.

\citeasnoun{herring07} suggest that computer mediated discussion is bounded both by technology and its affordances and the social context where the discussion takes place. She observes that neither of these is the determinant factor, rather that these factors are related to others.

\begin{table}
\caption{Suggested modalities}
\begin{tabular}{ll}
Anonymity & (Danet 1998, Donath 1999)  \\ 
Quoting & (Severinson-Eklundh \& Macdonald 1994) \\ 
Existence of social norm & \cite{sukumaran11} \\
\end{tabular} 
\label{tab:modalities}
\end{table}

\possessivecite{smith09} work on assessment of deliberative practices suggests, among others\footnote{Other aspects \possessivecite{smith09} suggest are transferability of the mechanics and impact of the citizens' participation. Naturally, they are important aspects and valuable further research questions.} that the both the quality of the discussion and diversity of participants should be evaluated. The quality of the discussion points to the area

Therefore the research questions must also cover both of these aspects:

\begin{enumerate}
\item How the design affects the use of the participation system?
\item How the design impacts the quality of the deliberation?
\end{enumerate}

\section{Approach}

In Table \ref{tab:modalities} I have identified different kind of conditions that may have an impact on the deliberation. Experimenting these conditions in an online forum will enable me to study deliberation in detail. Even while experimental methods in political science are not mainstream at this time \cite{green03,druckman06}, the need for comparing different situations and providing support for applied policy has been identified \cite{stoker10}. \citeasnoun{stoker10} discusses in great length about using the skills and the knowledge of political scientists to support the policy design ongoing by conducting field trials. Improving the online discussion forums may be one area of applying this kind of knowledge, and \citeasnoun[256]{wright12} identifies this research opportunity, suggesting that online forum research should focus on

\begin{quote}
\texttt{[- -]} comparatively testing different forum interfaces to see how they impact deliberation (and other values) would 
enhance Saward’s democratic toolkit.
\end{quote}

Evaluating the users of online forums is not a novel research question. Classical methods apply statistical methods, such as logistic regression allows detailed examination of several factors affecting users behaviour together. Previous research into participation suggest that participation is unequal -- that there are differences based on e.g socio-economic factors. In the most classical form of participation, voting \citeaffixed{lijphart97}{e.g.}. Similarly, in study of e-participation has observed that certain groups are more active to participate \cite{baek11,albrecht06,strandberg08}.

\textbf{TODO}: details on these results and set hypothesis

For measuring the deliberative qualities of the discussion, one potential approach is to apply \possessivecite{steenbergen03} Discourse Quality Index (DQI). The benefit of this measurement tool is its strict approach to measurement and empirical use in the field of political science. At the same time \possessivecite{steenbergen03} approach is locked to a certain view of deliberative democracy, which is criticized by \citeasnoun{freelon10} -- however his proposed framework lacks empirical measuring guides and therefore could lead to results not comparable with other research. This is why I am keen on using more empirically grounded tool, such as the DQI.

An interesting view in \citeasnoun{baek11} is their approach also to consider the experience of the participants as part of deliberation.

\textbf{TODO}: details on the method and few examples of applications

\bibliographystyle{apsr}
\bibliography{proposal}

\end{document}