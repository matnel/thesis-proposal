\documentclass[journal,a4paper]{IEEEtran}
%\documentclass{article}
\usepackage[utf8]{inputenc}
\usepackage{hyperref}

\usepackage{harvard}
\citationmode{abbr}

\author{Matti Nelimarkka}
\title{Draft: Designed for Democracy}
\begin{document}

\maketitle

%% nice output
\setlength{\parindent}{0pt}
\setlength{\parskip}{1ex}


\section{Introduction}

The impact of the emerging technologies to the democracy has been widely studied topic. The rise of the Internet is again one media that provides potential for support of democracy and political participation. The use of the Internet has been studied to support the institutional participation, to support novel forms of social movements and to facilitate discussions, among others. Looking the discussion facilitation\footnote{This choice reflects a deliberative view about democracy, however even in the field of e-democracy there are different positions about this \citeaffixed{dahlberg11}{e.g.}.} in detail and focusing deliberation in democratic process, different platforms have been experimented \citeaffixed{macintosh02,jen03}{e.g.} and evaluated \citeaffixed{strandberg08,dahlberg01,albrecht06,chadwick03}{e.g.}.

Considering the recent developments in studying online discussion, \citeasnoun{graham12} suggest that the methodology can be extended beyond classical political discussion forums to other kind of discussions that facilitate social interaction. This observation, that I agree on, indicates also the societal impact of detail examination of online discussion forums: they cover more than the classical institutional politics: in \possessivecite{graham12} case, the focus of study was discussion related to a television program.

Based on the extensive research it is not clear weather the Internet supports deliberative democracy or not. \citeasnoun[142]{smith09} concludes that "whilst there have been staggering developments in the commercial work, the potential for using ICT to increase and deepen citizen participation in political decision-making has legged somewhat behind."

TODO: should discuss online democracy in general? I.e. difference between online and offline deliberation?

\section{Research questions}

This research proposal aims to study the impact design decisions have on participation. The suggestion is that certain decisions, such as anonymity and choices on moderation affect quality of deliberation. Previous research in this domain is narrow and does not have a strong empirical focus. \citeasnoun{fatland07} has designed a new interface for political participation from deliberative point of view, but it has not been evaluated in practice. \citeasnoun{wright07} studied the impact of the moderation for the discussion. They suggest that moderation helps to keep the discussion on topic. This study however does not aim to explain the impact of other aspects, such as anonymity and profiles, to deliberation quality nor the quality of the deliberation been examined in detail.

\citeasnoun{wright07} however emphasize the need to study the process and use of online forums, especially as there may be power relations in the design process. Certain steps have been made to progress in this field. \citeasnoun{jen03} and nn discuss specially the role of politicians as part of participation, and xx and yy discuss about the administration

TODO: Here, define a set of conditions based on existing research?

\possessivecite{smith09} work on assessment of deliberative practices suggests, among others\footnote{Other aspects \possessivecite{smith09} suggest are transferability of the mechanics and impact of the citizens' participation. Naturally, they are important aspects and valuable further research questions.} that the both the quality of the discussion and diversity of participants should be evaluated. The quality of the discussion points to the area

Therefore the research questions must also cover both of these aspects:

\begin{enumerate}
\item How the design affects the use of the participation system?
\item How the design impacts the quality of the deliberation?
\end{enumerate}

\section{Approach}

Above I've identified different kind of conditions that may have an impact on the deliberation. Experimenting these conditions in an online forum will enable me to study deliberation in detail. Even while experimental methods in political science are not mainstream at this time \cite{green03,druckman06}, the need for comparing different situations and providing support for applied policy has been identified \cite{stoker10}. \citeasnoun{stoker10} discusses in great length about using the skills and the knowledge of political scientists to support the policy design ongoing by conducting field trials. Improving the online discussion forums may be one area of applying this kind of knowledge, and \citeasnoun[256]{wright12} identifies this research opportunity, suggesting that online forum research should focus on

\begin{quote}
\texttt{[- -]} comparatively testing different forum interfaces to see how they impact deliberation (and other values) would 
enhance Saward’s democratic toolkit.
\end{quote}

Evaluating the users of online forums is not a novel research question. Classical methods apply statistical methods, such as logistic regression allows detailed examination of several factors affecting users behaviour together. Previous research into participation suggest that participation is unequal -- that there are differences based on e.g socio-economic factors. In the most classical form of participation, voting \citeaffixed{lijphart97}{e.g.}. Similarly, in study of e-participation has observed that certain groups are more active to participate \cite{albrecht06,strandberg08}.

TODO: details on these results

For measuring the deliberative qualities of the discussion, one potential approach is to apply \possessivecite{steenbergen03} Discourse Quality Index (DQI).

TODO: details on the method and few examples of applications

\bibliographystyle{apsr}
\bibliography{proposal}

\end{document}