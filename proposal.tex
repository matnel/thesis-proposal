\documentclass[journal,a4paper]{IEEEtran}
%\documentclass{article}
\usepackage[utf8]{inputenc}
\usepackage{hyperref}

\usepackage{harvard}
\citationmode{abbr}

\author{Matti Nelimarkka}
\title{Draft: Designed for Deliberative Discussion}
\begin{document}

\maketitle

%% nice output
\setlength{\parindent}{0pt}
\setlength{\parskip}{1ex}

\textit{Note: Last updated after Kai Huotari's email.}

\section{Introduction}

The impact of the emerging technologies to the democracy has been a widely studied topic. The rise of the Internet is again one media that provides potential to support democracy and political participation, and it has been studied how the use of the Internet supports the institutional participation, novel forms of social movements and to facilitates discussions, among others. Looking the discussion facilitation\footnote{This choice reflects a deliberative view about democracy, however even in the field of e-democracy there are different positions about this \citeaffixed{dahlberg11}{e.g.}.} in detail and focusing deliberation in decision making process, different platforms have been experimented \citeaffixed{macintosh02,jen03}{e.g.} and evaluated \citeaffixed{strandberg08,dahlberg01,albrecht06,chadwick03}{e.g.}. However, scholars have not been focusing on the design aspects of these forums, especially implications of the design to the deliberation. This point of view is presented and argued in this proposal.

I also argue that this proposal has not only scientific importance but social value. Online participation has extended beyond classical political discussion forums to forums that support all kind of social interaction \cite{graham12}, for example in \possessivecite{graham12} case, discussion related to a television program was examined from deliberative point of view.

\textbf{TODO: Vaikuttaa irtonaiselta} I naturally accept that after the extensive research, it is not clear weather the Internet can support deliberative democracy. \citeasnoun[142]{smith09} concludes that "whilst there have been staggering developments in the commercial work, the potential for using ICT to increase and deepen citizen participation in political decision-making has legged somewhat behind."

\textbf{TODO}: should discuss (shortly) of online democracy in general? I.e. difference between online and offline deliberation?

\section{Research questions}

This research proposal aims to study how design choices impact the quantity and quality of online forum's participation. When designing such forum, several design choices -- some of which are listed on Table \ref{tab:modalities} -- are made. These choices may affect the use of the forum, however in the field of political science this kind of empirical and applied focus is still missing \cite[256]{wright12}.

However, \citeasnoun{wright07} arguest that the process side of online forums should be studied further. Certain steps have been made to progress into this direction. For example, \citeasnoun{jen03} and nn discuss specially the role of politicians as part of participation, and xx and yy discuss about administration's role.

Research on forums design includes \possessivecite{fatland07} novel interface for deliberative political participation -- but it has not been evaluated. Empirical evaluations are also lacking, for example \citeasnoun{wright07} studied the impact of moderation. They suggest that moderation helps to keep the discussion on topic, but does not discuss other factors and aspect affecting the participation, such as anonymity and profiles. And the quality of the deliberation has not been examined in detail.

\textbf{TODO}: Based on Vili's and Kai's observation, look deeper into other fields.

\begin{table}
\caption{Suggested modalities}
\begin{tabular}{ll}
Feature x & (Meikäläinen, xxxx)  \\ 
 & \textbf{TODO}: Here, define a set of conditions based on existing research?  \\ 
\end{tabular} 
\label{tab:modalities}
\end{table}

\possessivecite{smith09} work on assessment of deliberative practices suggests, among others\footnote{Other aspects \possessivecite{smith09} suggest are transferability of the mechanics and impact of the citizens' participation. Naturally, they are important aspects and valuable further research questions.} that the both the quality of the discussion and diversity of participants should be evaluated. The quality of the discussion points to the area

Therefore the research questions must also cover both of these aspects:

\begin{enumerate}
\item How the design affects the use of the participation system?
\item How the design impacts the quality of the deliberation?
\end{enumerate}

\section{Approach}

In Table \ref{tab:modalities} I have identified different kind of conditions that may have an impact on the deliberation. Experimenting these conditions in an online forum will enable me to study deliberation in detail. Even while experimental methods in political science are not mainstream at this time \cite{green03,druckman06}, the need for comparing different situations and providing support for applied policy has been identified \cite{stoker10}. \citeasnoun{stoker10} discusses in great length about using the skills and the knowledge of political scientists to support the policy design ongoing by conducting field trials. Improving the online discussion forums may be one area of applying this kind of knowledge, and \citeasnoun[256]{wright12} identifies this research opportunity, suggesting that online forum research should focus on

\begin{quote}
\texttt{[- -]} comparatively testing different forum interfaces to see how they impact deliberation (and other values) would 
enhance Saward’s democratic toolkit.
\end{quote}

Evaluating the users of online forums is not a novel research question. Classical methods apply statistical methods, such as logistic regression allows detailed examination of several factors affecting users behaviour together. Previous research into participation suggest that participation is unequal -- that there are differences based on e.g socio-economic factors. In the most classical form of participation, voting \citeaffixed{lijphart97}{e.g.}. Similarly, in study of e-participation has observed that certain groups are more active to participate \cite{albrecht06,strandberg08}.

\textbf{TODO}: details on these results and set hypothesis

For measuring the deliberative qualities of the discussion, one potential approach is to apply \possessivecite{steenbergen03} Discourse Quality Index (DQI).

\textbf{TODO}: details on the method and few examples of applications

\bibliographystyle{apsr}
\bibliography{proposal}

\end{document}