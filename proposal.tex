\documentclass[journal,a4paper]{IEEEtran}
%\documentclass{article}
\usepackage[utf8]{inputenc}
\usepackage{hyperref}

\usepackage{harvard}
\citationmode{abbr}

\author{Matti Nelimarkka}
\title{Draft: Supporting Deliberative Discussion with Enhanced Design}

\begin{document}

\maketitle

%% nice output
\setlength{\parindent}{0pt}
\setlength{\parskip}{1ex}

\begin{abstract}
This research proposal suggests using experimental approach to study online participation. By developing and studying the use of software prototypes, the factors impacting online participation can be examined in detail, and by combining both political science and human-computer interaction, contribute in a novel way.

The first research question focuses on \textit{the quantity of participation}. The focus should be both in the uptake and long term use of the system. The research tradition in this area has focused in examining impact of background variables into participation and that is the aim of mine too. This should take in account not only the uptake, but level of participation and the impact of long-term use into the system.

The second research question aims to study \textit{the quality of participation}. This should take account both the amore objective quality -- which can be estimated with \possessivecite{steenbergen03}  discourse quality index (DQI) -- and the subjective experience of participation. Here the traditions are mixed, the first research aims to quantify qualitative material and use statistical methods to understand it, where as the second method is more classical qualitative study, aiming to understand how participants feel about the participation, e.g. via interviews.
\end{abstract}

\section{Introduction}
Scholars of political science have suggested that the decisions made in our society could be improved by arguing and discussing them before hand \citeaffixed{habermas87,fiskin91,cohen97}{e.g.}. This approach can be called deliberative democracy and the concept has been studied in detail both by the normative political scientists and the empirical political scientists. Deliberative democracy requires a venue for discussion, which may be a town hall meeting, mini-publics or, thanks to the rise of the Internet, an online forum \citeaffixed{smith09}{e.g.}.

Online participation is a widely studied topic and discussion facilitation is just one of the fields in focus \citeaffixed{dahlberg11}{c.f.}. The research has focused on studying how political discussion takes place in venues of different kind \citeaffixed{macintosh02,jen03}{e.g.} and evaluated these discussions \citeaffixed{strandberg08,dahlberg01,albrecht06,chadwick03}{e.g.}. Recently, research has extended also in non-political forums as a place for discussion \cite{graham12}. This extension indicates the social importance to study the online discussion forms in detail, as they are not only in focus of politics, but in general communication. However, less scholarly focus has been on the design of the discussion forums and the implications of design \citeaffixed{wright07,sukumaran11}{however, e.g.}.

This proposal suggests that these design choices and their impact on the user behaviour should be further studied in an experimental setting to understand how to design these forums in an improved way. This is a natural place to combine skills and methods from human-computer interaction (HCI), especially in the area of experimental design and field trials \citeaffixed{brown11}{e.g.}.

\section{Research questions}
As noted above, the design of an online forum is not extensively studied. However, scholars such as \citeasnoun{wright12}, argue that the process side should be further studied.  Research on forums design includes \possessivecite{fatland07} novel interface for deliberative political participation -- but it has not been evaluated. Also the more empirical evaluations are lacking, for example \possessivecite{wright07} study on the impact of moderation. They suggest that moderation helps to keep the discussion on the topic, but does not discuss other factors and aspect affecting the participation, such as anonymity and profiles. The quality of the deliberation has not been examined in detail.

Naturally, computer mediated discussions are more than the technical features of the systems. \citeasnoun{herring07} suggests that computer mediated discussion is bounded both by technology and its affordances and the social context where the discussion takes place. She continues that neither of these is the determinant factor, rather that these factors are related to others. The role of social contexts is also studied, indicating they have a big impact on the discussion \cite{sukumaran11,underhill03}.

Therefore, the system design should take in account both the social context of use and the technical affordances the system provides. Examples of design choices impacting the social context could be a choice of moderation strategy, where as more technical choices could be a choice if the system uses pseudonyms or real identities. As we can from these examples see, these factors can be strongly linked together: allowing the system to be anonymous naturally impacts the social context of use, even while it is more a technical factor.

The exact research questions suggested in this proposal combine the two aspects of participation: quantity and quality \citeaffixed{smith09}{e.g.}. They explore how the system design -- considered in broad sense, as explained above -- impacts the participation.

\begin{enumerate}
\item How does the design affect the use of the participation system?
\item How does the design impact the quality of the deliberation?
\end{enumerate}

\section{Approach}
This research proposal aims to study how design choices impact the deliberativeness of online forum. As suggested by \citeasnoun{smith09}, the quality of deliberation can be determined by examining who participate, how they participate and what is the impact of participation. Measurements of these will be the dependent variables. There are a variety of potential interventions that can be made, and they will be a major part of the experimental design. Examples of potential interventions are listed in Table \ref{tab:modalities}.

\begin{table}
\caption{Possible modalities and related research}
\begin{tabular}{ll}
Identity & \cite{danet98,donath99}  \\ 
Quoting & \cite{eklundh94} \\ 
Social norms and context & \cite{sukumaran11,underhill03} \\
\end{tabular} 
\label{tab:modalities}
\end{table}

Experimenting these conditions in an online forum will enable me to study deliberation and impact of the design choices in detail. Even while experimental methods in political science are not mainstream at this time \cite{green03,druckman06}, the need for comparing different situations and providing support for applied policy has been identified \cite{stoker10}. \citeasnoun{stoker10} discusses in great length about using the skills and the knowledge of political scientists to support the policy design ongoing by conducting field trials. Improving the online discussion forums may be one area of applying this kind of knowledge, and \citeasnoun[256]{wright12} identifies this research opportunity, suggesting that online forum research should focus on

\begin{quote}
\texttt{[- -]} comparatively testing different forum interfaces to see how they impact deliberation (and other values) would 
enhance Saward’s democratic toolkit.
\end{quote}

Evaluating the users of online forums is not a novel research question. Classical methods apply statistical methods, such as logistic regression allows detailed examination of several factors affecting users' behaviour together. Previous research into participation suggest that participation is unequal -- that there are differences e.g  based on socio-economic factors, even in the most classical form of participation, voting \citeaffixed{lijphart97}{e.g.}. Similarly, in the study of e-participation has observed that certain groups are more active to participate \citeaffixed{baek11,albrecht06,strandberg08}{e.g.}.

For measuring the deliberative qualities of the discussion, one potential approach is to apply \possessivecite{steenbergen03} Discourse Quality Index (DQI). The benefit of this measurement tool is its strict approach to measurement and empirical use in the field of political science. At the same time, \possessivecite{steenbergen03} approach is locked to a certain view of deliberative democracy, which is criticized by \citeasnoun{freelon10} -- however his proposed framework lacks empirical measuring guides and therefore could lead to results not comparable with other research. This is why I am keen on using more empirically grounded tools, such as the DQI.

An interesting view in \citeasnoun{baek11} is their approach also to consider the experience of the participants as part of deliberation. They argue that the deliberation depends on the perceived heterogenity, not only about the true heterogenity of the population. Even while they do not make the clear link, I see the percived experience related to the social context discussed above. Also, in general focus on experience is seen important in modern HCI research, where the user experience is seen to impact product's success. Therefore, studies should also take account this kind of subjective feeling.

\bibliographystyle{apsr}
\bibliography{proposal}

\end{document}