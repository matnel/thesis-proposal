\documentclass[journal,a4paper]{IEEEtran}
\usepackage[utf8]{inputenc}
\usepackage{hyperref}

\usepackage{harvard}
\citationmode{abbr}

\author{Matti Nelimarkka}
\title{Designed for Democracy}
\begin{document}

\maketitle

%% nice output
\setlength{\parindent}{0pt}
\setlength{\parskip}{1ex}


\section{Introduction}

The impact of the emerging technologies to the democracy has been widely studied topic. The rise of the Internet is again one media that provides potential for support of democracy and political participation. The use of the Internet has been studied to support the institutional participation, to support novel forms of social movements and to facilitate discussions, among others. Looking the discussion facilitation in detail and focusing deliberation in democratic process, different platforms have been experimented \citeaffixed{macintosh02,jen03}{e.g.} and evaluated \citeaffixed{strandberg08,albrecht06}{e.g.}.

Considering the recent developments in studying online discussion, \citeasnoun{graham12} suggest that the methodology can be extended beyond classical political discussion forums to other kind of discussions that facilitate social interaction. This observation, that I agree on, indicates also the societal impact of detail examination of online discussion forums: they cover more than the classical institutional politics: in \citeasnoun{graham12} case, the focus of study was discussion related to a television program.

Based on the extensive research it is not clear weather the Internet supports deliberative democracy or not.

\section{Research questions}

This research proposal aims to study the impact design decisions have on participation. The suggestion is that certain decisions, such as anonymity and choices on moderation affect quality of deliberation. Previous research in this domain is narrow and does not have a strong empirical focus. \citeasnoun{fatland07} has designed a new interface for political participation, but it has not been evaluated in practice. \citeasnoun{wright07} Some research has focused on

\possessivecite{smith09} work on assessment of deliberative practices suggests, among others\footnote{Other aspects \possessivecite{smith09} suggest are transferability of the institution and impact of the citizens' participation. Naturally, they are important aspects,} that the both the quality of the discussion and diversity of participants should be evaluated. The quality of the discussion points to the area

Therefore the research questions must also cover both of these aspects, and 

\begin{enumerate}
\item How the design of the online participation affects the users?
\item How the design of the online participation impacts the quality of the discussion?
\end{enumerate}

\section{Approach}

Use of field experimentation is required ...

For measuring the deliberative qualities of the discussion, one potential approach is to apply \possessivecite{steenbergen03} Discourse Quality Index (DQI).

\bibliographystyle{apsr}
\bibliography{proposal}

\end{document}